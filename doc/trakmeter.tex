\input{include/settings.sty}
\input{include/hyphenation.sty}

\title{traKmeter}
\author{Martin Zuther}

\begin{document}

\title{traKmeter}

\subtitle{
  \normalsize{\textrm{\textmd{
        \vfill
        Loudness meter for correctly setting up \\
        tracking and mixing levels
        \vfill
        \vspace{1.5em}
        \begin{figure}[H]
          \centering{}
          \includegraphics[scale=0.275,clip]{include/images/kmeter.png}
        \end{figure}
        \vfill
      }}}
}

\author{\normalsize\copyright{} 2012
  \href{http://www.mzuther.de/}{Martin Zuther}}

\date{\normalsize \emph{Last edited on \today}}

\maketitle

\tableofcontents

\clearpage  % layout

\chapter{Digital recordings}
\label{chap:digital_recordings}

The digital revolution brought a lot of advantages to the field of
audio processing such as higher fidelity, less noise and non-degrading
copies.  Unfortunately, however, digital audio also introduced some
problems of its own.

Whereas the analog domain is relatively inert against levels exceeding
\SI{0}{\dBFS} (overdriving some analog equipment actually sounds
good), the digital domain punishes even small transgressions into
forbidden territory with harsh clipping.

And while digital audio can be transferred without loss in quality, it
is degraded by each and every calculation, be it a simple change in
level, equalisation or a fancy effect.  Crossing domains from analog
to digital and \emph{vice versa} leads to additional degradation.
Finally, changes in bit depth and sample rate, jitter and inter-sample
peaks are nothing for the weak of heart.

However, most of these obstacles can be overcome easily by proper gain
staging, minimising the crossing of domains and choosing appropriate
bit depth and sample rates.  If you also learn how to properly test
and operate your equipment, you're well on your way to pure audio
bliss!

\section{Gain staging}
\label{sec:gain_staging}

Professional analog audio equipment is designed to be run at a nominal
level of \textbf{\SI[addsign=all]{+4}{\dBu}} (\SI{1.23}{\VRMS})
leaving a headroom for peaks of about \SI{20}{\dB}.  This in turn is
consistent with the maximum crest factor of analog audio signals.

Thus, driving all analog audio equipment at \SI[addsign=all]{+4}{\dBu}
ensures an optimal signal-to-noise ratio while preventing clipping and
keeping transients intact.  The process of setting audio devices to
run at optimal input and output levels is called gain staging.

Now let's transfer gain staging to the digital domain.  The maximum
crest factor of analog audio signals is around \SI{20}{\dB}, so we'll
have to adjust the headroom accordingly.  To keep with international
standards, we'll follow the
\href{http://tech.ebu.ch/publications/r068}{EBU R68-2000}
recommendation and set our average input and output levels to
\textbf{\SI{-18}{\dBFS} RMS}.

Again, this ensures an optimal signal-to-noise ratio while preventing
clipping.  It also drives (most of) your digital audio equipment and
plug-ins at the ``sweet spot''.

Another recommendation is that peak levels should not exceed
\textbf{\SI{-9}{\dBFS}} (EBU R68-2000) during recording, leaving
enough space for inter-sample peaks.  Inter-sample peaks are audio
peaks that lie \emph{in between} two successive samples and may lead
to unpredictable clippings during digital-to-analog conversion.

\section{Digital audio myths}
\label{sec:digital_audio_myths}

I can almost hear you: you have heard that digital recordings should
be performed at peak levels close to but not exceeding \SI{0}{\dBFS}
(``digital full-scale'').  Heck, this misinformation has ended up in
the manuals of some professional audio equipment.  But for the reasons
given above it is plain wrong.

Just do the math for a bit depth of 16 bit: even if your recordings
peak at a level of \SI{-18}{\dBFS} and you discard the least
significant bit (some people claim that it mostly consists of errors),
this still leaves you with a signal-to-noise ratio of \SI{72}{\dB}.
That is about what you can expect from the best professional analog
tape machines and recording desks!

\begin{wrapfigure}{r}{0pt}
  \includegraphics[scale=0.24,clip]{include/images/otari_meter.png}
\end{wrapfigure}

As some sort of proof, the analog inputs and outputs of my 16-bit hard
disk recorder (Otari PD-80) are aligned to
``\SI[addsign=all]{+4}{\dBu} (\SI{-15}{\dB} from digital
full-scale)''.  The manufacturer has even marked this level on the
meter bridge (small triangle on the photo).  Although I admit that the
mark is only useful for audio alignment, given that it sits on a peak
meter\ldots

If you do not believe me still, simply record a couple of tracks
peaking at \SI{-1}{\dBFS} and then repeat the procedure, this time
having the tracks peak at \SI{-9}{\dBFS}.  Mix the tracks, adjust
master levels and compare the mixes.  The mix from the tracks recorded
at higher level will almost certainly sound worse.

\section{Introducing traKmeter}
\label{sec:introducing_trakmeter}

Most digital audio equipment sadly only has peak meters.  This is
readily understandable as you want to avoid digital clippings by all
means.  However, the lack of average meters makes correct gain staging
almost impossible.

For gain staging, you need average meters or, even better, a
combination of peak and average meters.  And this is were
\textbf{traKmeter} comes in.

\chapter{traKmeter}
\label{chap:trakmeter}

\textbf{traKmeter} consists of two meters, a peak meter on top and an
average meter below.  Both meters have an area of green LEDs that is
enclosed by first yellow and then red LEDs.

You may have noticed that the average meter's green area is centred
around the \textbf{\SI{-18}{\dBFS} RMS} mark.  This number should be
vaguely familiar.  Remember, it corresponds to the optimal average
audio level in the digital domain.

A fully lit yellow LED on the top end of the peak meter corresponds to
a level of \textbf{\SI{-9}{\dBFS}}.  Again, this number should be
familiar.  Remember, peak levels in the digital domain shouldn't
exceed this level.

Thus, by keeping meter readout in the green areas and from entering
the yellow and red areas on top of each meter, you will automatically
record at optimal audio levels.

\newpage %% layout

\section{Recording with traKmeter}
\label{sec:recording_with_trakmeter}

Open up an instance of \textbf{traKmeter} and set it up so that it
measures your audio input.  That can be done either by starting the
standalone version and connecting it to one or more input channels of
your sound card, or by inserting a plug-in instance into an input
channel of your digital audio workstation.

In the second case, take care that your digital audio workstation
doesn't add additional headroom and that no processing takes place
before \textbf{traKmeter}.  This can be ascertained by feeding
calibration tones into your sound card or by directly comparing the
readouts of standalone and plug-in version.

Now, feed a signal you want to record into an audio input channel and
adjust the level using \textbf{traKmeter}.  Try to set the input level
so that transients reach \textbf{\SI{-18}{\dBFS} RMS} on the average
meter.  Make sure that the peak level never exceeds
\textbf{\SI{-9}{\dBFS}}.  In case both conditions cannot be met
simultaneously, adjust the peak level only.

\chapter{Recording tips}
\label{chap:recording_tips}

Over the years, I have accumulated a couple of recording tips.  You
may not know some of them, so read ahead!

\begin{description}

\item[Use a good preamp.  Crank it up.]  ``Good'' doesn't mean your
  preamp has to have a lot of channels or features.  In the contrary!
  Go for a simple design and invest your money in better quality
  instead.

  Crank up the preamp to yield the needed output level.  Do not fear
  the preamp's internal noise -- making up for low gain in later
  stages will result in even more noise!  Also see
  \ref{sec:gain_staging}.

\item[Avoid unballanced equipment.]  Run all signals on balanced lines
  with a nominal level of \SI[addsign=all]{+4}{\dBu}.  If you can't,
  use DI boxes or transformers and read the previous sentence again.

\item[Use short audio chains.]  All equipment adds noise or may
  otherwise degrade audio, so keep your audio chains as short as
  possible.

\item[Record at lower levels.]  Record digital audio at
  \textbf{\SI{-18}{\dBFS} RMS} with peak levels not exceeding
  \textbf{\SI{-9}{\dBFS}}.  For an in-depth explanation, see
  \ref{sec:gain_staging}.

\item[Record in mono.]  Most audio sources do not contain stereo
  information that is useful in a mixing context (notable exceptions
  are audience recordings, string sections and sometimes pianos).  The
  pseudo-stereo effects of some synthesisers may even cause phasing
  issues in the mixing stage.

  Recording these sources in stereo will only waste space and make you
  miserable during mixing.  So simply record them in mono.

\item[Use high bit depths.]  Do yourself a favour and record at 24 bit
  instead of 16 bit.  The additional bits provide an incredible amount
  of extra detail and you can record at lower levels without losing
  information.  When properly dithered, changing to a lower bit depth
  even preserves some of that detail.

  Also, if you edit audio files or apply effects, calculation errors
  are inevitable.  At 24 bit, however, these artifacts are
  \SI{48}{\dB} lower in level (and thus inaudible) compared to 16 bit
  audio files.

  Your digital audio workstation's bus should use at least 32 bits to
  avoid accumulation of the above-mentioned artifacts.

\item[Avoid sample rate conversion.]  Sample rate conversion usually
  degrades audio (especially small changes of a few \si{\kilo\hertz}),
  so try to record at the target sample rate.  For instance, tracking
  for a CD release should be carried out at \SI{44.1}{\kilo\hertz}
  instead of \SI{48}{\kilo\hertz}.

  There are of course exceptions to the rule, for instance you may
  prefer to track on a professional DAT machine (\SI{48}{\kilo\hertz})
  when your only other choice is using a consumer audio interface.

  For tracking at higher sample rates, it pays to use exact multiples
  of your target sample rate (such as \SI{88.2}{\kilo\hertz} instead
  of \SI{96}{\kilo\hertz}) if your hardware and software permit.
  Please note that some professionals actually advise against using
  higher sample rates due to the possible build-up of noise beyond
  \SI{20}{\kilo\hertz}.  It is also much more demanding on your
  computer, audio equipment and plug-ins -- and may not be worth it.
  Try changing from recording at 16 bit to 24 bit first!

  Finally, only use professional software for sample rate conversion.
  This is by no means a trivial task.

\item[Treat your room.]  TO DO!

\item[Experiment with microphone placement.]  TO DO!

\end{description}

\chapter{Installation}
\label{chap:installation}

In order to use the pre-compiled binaries, simply extract the
traKmeter files from the downloaded archive.  For the VST plug-in,
you'll then have to move the extracted files to your plug-in folder
(\path{~/.vst}, \path{C:\Program Files\Steinberg\VstPlugins\} or the
  like).

\chapter{Controls}
\label{chap:controls}

\textbf{TO DO!}

% \section{Reset button}

% \begin{wrapfigure}{r}{0.14\linewidth}
%   \includegraphics[scale=\screenshotscale,clip]{include/images/button_reset.png}
% \end{wrapfigure}

% Click on this button to reset all meters and peaks.  You can also get
% rid of graphical artifacts, because the meters will be redrawn as
% well.

% \section{Validation button}
% \label{sec:validation_button}

% \begin{wrapfigure}{r}{0.14\linewidth}
%   \includegraphics[scale=\screenshotscale,clip]{include/images/button_validate_on.png}
%   \newline \vspace{-0.9\baselineskip}
%   \includegraphics[scale=\screenshotscale,clip]{include/images/button_validate_off.png}
% \end{wrapfigure}

% Click on this button to open the \textbf{validation window} (see
% \ref{chap:validation}) which allows you to play an audio file (WAV,
% AIFF or FLAC) through traKmeter and dump internal data.  During
% validation, the button will light up and clicking it will stop the
% validation.

% On Linux, dumped data will be written to \path{stderr}, so just start
% the traKmeter standalone or your VST host from the shell and watch the
% output coming.  On other systems, have a look at your VST host's log
% files (I have successfully used Ableton Live for this).  If that
% doesn't work, you might have to start either the traKmeter standalone
% or your VST host from a debugger.

% As a side note, \textbf{SMA(50)} designates the simple moving average
% of 50 values, a neat way to emphasise trends and eliminate short-term
% fluctuations.

% \section{About button}

% \begin{wrapfigure}{r}{0.14\linewidth}
%   \includegraphics[scale=\screenshotscale,clip]{include/images/button_about_on.png}
%   \newline \vspace{-0.9\baselineskip}
%   \includegraphics[scale=\screenshotscale,clip]{include/images/button_about_off.png}
% \end{wrapfigure}

% Clicking on this button will open the \textbf{about window} where you
% will be informed about version number, contributors, copyright and the
% GNU General Public License.

% \section{Display license}

% \begin{wrapfigure}{r}{0.15\linewidth}
%   \includegraphics[scale=\screenshotscale,clip]{include/images/button_gpl_on.png}
%   \newline \vspace{-0.9\baselineskip}
%   \includegraphics[scale=\screenshotscale,clip]{include/images/button_gpl_off.png}
% \end{wrapfigure}

% This button is located in the \textbf{about window} and does not only
% advertise that you are using free software licensed under the
% \textbf{GNU General Public License} -- when clicked, it will also open
% the license's website in your web browser \dots

\chapter{Meters}
\label{chap:meters}

\section{Average level meter}

The average level meter uses an averaging period of \num{1024}
samples.  This meter exhibits a completely flat frequency response.

On rising levels, it takes \SI{10}{\milli\second} for the meter to
reach \SI{99}{\percent} of the final reading.  When above
\SI{-22}{\dBFS}, levels fall with a speed of \SI{6}{\dB\per\second}.
Ballistics then change to \SI{300}{\milli\second} for the meter to
fall to \SI{99}{\percent} of the final reading.

Peak levels will be held for \SI{10}{\second} and then fall with a
speed of \SI{8.67}{\dB\per\second}.

\section{Peak level meter}

The peak level meter also possesses a completely flat frequency
response.  It has a rise time of one sample and a fall time of
\SI{8.67}{\dB\per\second}.

Peak levels will be held until the meter is reset.

\chapter{Validation}
\label{chap:validation}

\textbf{TO DO!}

% I have gone to great lengths to ensure that all meters read correctly.
% You want to validate for yourself?  Just download and extract the
% source code.  The directory \path{validation} contains instructions
% and FLAC-compressed wave files.  To validate \textbf{ITU-R} mode,
% please download \href{http://www.itu.int/pub/R-REP-BS.2217}{ITU-R
%   BS.2217} and follow the instructions (at the time being, the tests
% for loudness gating should be ignored).  A word of warning: these
% audio files may \textbf{damage your ears} and speakers, so please
% watch your monitor levels!

% \begin{wrapfigure}{r}{0.395\linewidth}
%   \includegraphics[scale=\screenshotscale,clip]{include/images/dialog_validation.png}
% \end{wrapfigure}

% After opening the \textbf{validation window} (see
% \ref{sec:validation_button}), click on the ellipsis button (the one
% with the dots) to select an audio file for playback through traKmeter.
% Please make sure that the sample rates of your host (\textbf{Host SR})
% and the audio file match, otherwise the results will not be correct.

% Now, select which \textbf{variables} (if any) should be dumped.  You
% may also restrict dumped data to a specific audio \textbf{channel}.

% Finally, click on the \textbf{validate} button to reset all meters and
% start playback of the selected audio file.  All audio input will be
% discarded during playback and for an additional ten seconds.  To stop
% playback early, simply click on the \textbf{validate} button again.

% In case you want to calibrate your monitor system, head over to
% \href{http://www.digido.com/media/downloads.html}{Bob Katz's download
%   section}, get the file labelled \textbf{\SI{-20}{\dBFS} RMS pink
%   noise stereo \num{44.1}}, set K-meter to \textbf{RMS} mode and click
% on the \textbf{validate} button.  Please ensure that all intermediate
% software and hardware mixers are set to the correct levels.

% \section{Validation status}

% \begin{minipage}{1.0\linewidth}
%   \renewcommand{\thempfootnote}{\arabic{mpfootnote}}
%   \begin{tabular}{>{\bfseries}llcc}

%     &
%     \textbf{Readout} &
%     \textbf{RMS} &
%     \textbf{ITU-R} \\

%     Avg level meter &
%     meter ballistics &
%     \Checkmark{} &
%     \Checkmark{} \\

%     &
%     readings &
%     \Checkmark{} &
%     --- \\

%     &
%     frequency response &
%     \Checkmark{} &
%     \Checkmark{} \\

%     &
%     pink noise &
%     \Checkmark{} &
%     --- \\

%     &
%     ITU-R BS.2217 &
%     --- &
%     \Checkmark{} \\

%     Peak level meter &
%     meter ballistics &
%     \Checkmark{} &
%     \Checkmark{} \\

%     &
%     readings &
%     \Checkmark{} &
%     \Checkmark{} \\

%     Maximum peak &
%     readings &
%     \Checkmark{} &
%     \Checkmark{} \\

%     Overload counter &
%     readings &
%     \Checkmark{} &
%     \Checkmark{} \\

%     Phase correlation &
%     readings &
%     \Checkmark{} &
%     \Checkmark{} \\

%     Stereo meter &
%     readings &
%     \Checkmark{} &
%     \Checkmark{} \\

%   \end{tabular}
% \end{minipage}

% \newpage %% layout

% \section{Frequency and phase response}

% Frequency and phase response have been determined at a sample rate of
% \SI{192}{\kilo\hertz} using
% \href{http://www.savioursofsoul.de/Christian/programs/measurement-programs/}{VST
%   Plugin Analyser}.

% \textbf{Frequency response of complete effect path (\SI{5}{\hertz} to
%   \SI{96}{\kilo\hertz}, \SI{0}{\dB} \SI{\pm 0.1}{\dB}):}

% \begin{center}
%   \includegraphics[scale=0.65,clip]{include/images/fft_192khz-freq-fx_path.png}
% \end{center}

% \textbf{Phase response of complete effect path (\SI{5}{\hertz} to
%   \SI{96}{\kilo\hertz}, \SI{0}{\degree}\,\SI{\pm 0.1}{\degree}):}

% \begin{center}
%   \includegraphics[scale=0.65,clip]{include/images/fft_192khz-phase-fx_path.png}
% \end{center}

% \newpage %% layout

% \textbf{Frequency response of band-limited RMS detection stage
%   (\SI{5}{\hertz} to \SI{96}{\kilo\hertz}, \SI{-140}{\dB} to
%   \SI{5}{\dB}):}

% \begin{center}
%   \includegraphics[scale=0.65,clip]{include/images/fft_192khz-freq-rms.png}
% \end{center}

% \textbf{Phase response of band-limited RMS detection stage
%   (\SI{5}{\hertz} to \SI{96}{\kilo\hertz}, \SI{-180}{\degree} to
%   \SI[addsign]{+180}{\degree}):}

% \begin{center}
%   \includegraphics[scale=0.65,clip]{include/images/fft_192khz-phase-rms.png}
% \end{center}
% \newpage %% layout

% \textbf{Frequency response of band-limited ITU-R BS.1770-1 detection stage
%   (\SI{5}{\hertz} to \SI{96}{\kilo\hertz}, \SI{-140}{\dB} to
%   \SI{5}{\dB}):}

% \begin{center}
%   \includegraphics[scale=0.65,clip]{include/images/fft_192khz-freq-itu_r.png}
% \end{center}

% \textbf{Phase response of band-limited ITU-R BS.1770-1 detection stage
%   (\SI{5}{\hertz} to \SI{96}{\kilo\hertz}, \SI{-180}{\degree} to
%   \SI[addsign]{+180}{\degree}):}

% \begin{center}
%   \includegraphics[scale=0.65,clip]{include/images/fft_192khz-phase-itu_r.png}
% \end{center}

% \newpage %% layout

% \textbf{Frequency response of band-limited RMS detection stage
%   (\SI{5}{\hertz} to \SI{96}{\kilo\hertz}, \SI{0}{\dB} \SI{\pm
%     1}{\dB}):}

% \begin{center}
%   \includegraphics[scale=0.65,clip]{include/images/fft_192khz-freq_zoomed-rms.png}
% \end{center}

% \textbf{Frequency response of band-limited ITU-R BS.1770-1 detection stage
%   (\SI{5}{\hertz} to \SI{96}{\kilo\hertz}, \SI{0}{\dB} \SI{-6}{\dB} to
%   \SI{4}{\dB}):}

% \begin{center}
%   \includegraphics[scale=0.65,clip]{include/images/fft_192khz-freq_zoomed-itu_r.png}
% \end{center}

% \chapter{Help needed}
% \label{chap:help_needed}

% As traKmeter was coded using cross-platform code, it should be easy to
% compile versions for Windows (\num{64} bit) and Mac OS X.  I just
% don’t have the adequate systems and compilers.

% In case you want to help, please see the next chapter for an email
% address.  You’ll need sufficient experience in coding, compiling and
% debugging, though, so no beginners please!

\chapter{Final words}
\label{chap:final_words}

I want to thank \textbf{Rickard} of Interfearing Sounds for asking me
how to use K-Meter for tracking.  This question and the following
thoughts really got traKmeter started.  I'd like to thank
\textbf{bram@smartelectronix} for his code to calculate logarithmic
rise and fall times.  I must also thank the \textbf{beta testers} and
\textbf{users of traKmeter} for sending kind words, suggestions and
bug reports.  Finally, I want to thank the \textbf{open source
  community} for making all of this possible.

Although coding traKmeter has been a lot of fun, it has also been a
lot of work.  So if you like traKmeter, why not send me a short email
and tell me so?  Write a few words about yourself, send suggestions
for future updates or volunteer to create a nice theme -- do whatever
you like!

Here is my email address (please remove ``\texttt{-nospam}''):

\begin{center}
  \texttt{"Martin Zuther" <code-nospam@mzuther.de>}
\end{center}

Thanks for using free software.  I hope you'll enjoy it!

\emph{VST is a trademark of Steinberg Media Technologies GmbH.  ASIO
  is a trademark and software of Steinberg Media Technologies GmbH.}

\appendix

\chapter{How to build traKmeter}
\label{chap:build_trakmeter}

\section{Preparing GNU/Linux}

To build traKmeter yourself, I recommend setting up a \texttt{chroot}
environment.  This is fast and easy to do on Debian-based systems and
might save you a \textbf{lot} of trouble.  At the time of writing, I'm
using Linux Mint 13 (Maya), but the procedure should be similar on
your distribution of choice.  If you aim at generic \num{64}-bit
compilation, simply change \texttt{i386} to \texttt{amd64}.

To install the necessary packages and install the \texttt{chroot} base
system, execute the following statements (please change
\path{http://ftp.de.debian.org/debian/} to a
\href{http://www.debian.org/mirror/list}{mirror} close to you):

\begin{verbatim}
  sudo apt-get install debootstrap schroot

  sudo mkdir -p /srv/chroot/squeeze_i386
  sudo debootstrap --variant=buildd \
    --arch i386 squeeze \
    /srv/chroot/squeeze_i386 \
    http://ftp.de.debian.org/debian/
\end{verbatim}

Running \path{debootstrap} will take some time.  Meanwhile, add the
following lines to \path{/etc/schroot/schroot.conf} (make sure you
remove all preceding white space so that each line begins in the first
column):

\begin{verbatim}
  [squeeze-i386]
  description=Debian 6 (Squeeze, i386)
  directory=/srv/chroot/squeeze_i386
  personality=linux
  root-users=username
  type=directory
  users=username,another_user
\end{verbatim}

Please make the necessary changes to \texttt{username}.  You may also
add additional users, like \texttt{another\_user}.  In case you are
setting up a \num{32}-bit \texttt{chroot} environment on a
\num{64}-bit system, you'll also have to change \texttt{linux} to
\texttt{linux32}.

When \path{debootstrap} is done, log in as superuser:

\begin{verbatim}
  schroot -c squeeze-i386 -u root
\end{verbatim}

to install a few packages.  The packages \path{less} and \path{vim}
are optional, but might come in handy:

\begin{verbatim}
  apt-get update
  apt-get -y install bash-completion libasound2-dev \
    libjack-jackd2-dev mesa-common-dev xorg-dev
  apt-get -y install less vim
  apt-get clean
\end{verbatim}

If you like \path{bash} completion, you might also want to open the
file \path{/etc/bash.bashrc} and unquote these lines:

\begin{verbatim}
  # enable bash completion in interactive shells
  [two more lines...]
  fi
\end{verbatim}

Finally, log out and log in as normal user:

\begin{verbatim}
  schroot -c squeeze-i386
\end{verbatim}

Congratulations -- after you have installed the dependencies (see
below), you are ready to build traKmeter!

\section{Dependencies}

\subsection{premake4}

\begin{tabbing}
  \hspace*{6em}\=\=\kill

  Importance:  \> required \\
  Version:     \> 4.3 \\
  License:     \> BSD \\
  Homepage:    \> \href{http://industriousone.com/premake}{industriousone.com/premake}
\end{tabbing}

\subsubsection{Installation}

Place the binary somewhere in your \path{PATH}.  Depending on your
platform, you should run \emph{premake} using the scripts
\path{build/run_premake.sh} or \path{build/run_premake.bat}.

\subsection{JUCE library}

\begin{tabbing}
  \hspace*{6em}\=\=\kill

  Importance:  \> required \\
  Version:     \> 1.53 \\
  License:     \> GPL v2 \\
  Homepage:    \> \href{http://www.rawmaterialsoftware.com/juce.php}{www.rawmaterialsoftware.com/juce.php}
\end{tabbing}

\subsubsection{Installation}

Extract the archive into the directory \path{libraries/juce}.

\subsection{Virtual Studio Technology SDK}

\begin{tabbing}
  \hspace*{6em}\=\=\kill

  Importance:  \> optional \\
  Version:     \> 2.4 \\
  License:     \> proprietary \\
  Homepage:    \> \href{http://ygrabit.steinberg.de/}{ygrabit.steinberg.de}
\end{tabbing}

\subsubsection{Installation}

Just extract the archive into the directory
\path{libraries/vstsdk2.4}.

\subsection{Audio Streaming Input Output SDK}

\begin{tabbing}
  \hspace*{6em}\=\=\kill

  Importance:  \> optional \\
  Version:     \> 2.2 \\
  License:     \> proprietary \\
  Homepage:    \> \href{http://ygrabit.steinberg.de/}{ygrabit.steinberg.de}
\end{tabbing}

\subsubsection{Installation}

Simply extract the archive into the directory
\path{libraries/asiosdk2.2}.

\subsection{Artistic Style}

\begin{tabbing}
  \hspace*{6em}\=\=\kill

  Importance:  \> optional \\
  Version:     \> 2.01 \\
  License:     \> LGPL v3 \\
  Homepage:    \> \href{http://astyle.sourceforge.net/}{astyle.sourceforge.net}
\end{tabbing}

This application formats the code so it looks more beautiful and
consistent.  Thus, you only have to install it if you plan to help me
with coding traKmeter.

\subsubsection{Installation}

Place the binary somewhere in your \path{PATH}.  Depending on your
platform, you should run \emph{astyle} using the scripts
\path{src/format_code.sh} or \path{src/format_code.bat}.

\section{Building on GNU/Linux}

After preparing the dependencies, start your \texttt{chroot}
environment, change into the directory \path{build} and execute

\begin{verbatim}
  ./run_premake.sh
  make config=CFG TARGET
\end{verbatim}

where \application{CFG} is one of \application{debug32},
\application{debug64}, \application{release32} and
\application{release64}, and \application{TARGET} is one of
\application{linux\_standalone\_stereo},
\application{linux\_standalone\_multi},
\application{linux\_vst\_stereo} and
\application{linux\_vst\_multi}.

The compiled binaries will end up in the directory \path{bin}.

\section{Building on Microsoft Windows}

After preparing the dependencies, change into the directory
\path{build} and execute

\begin{verbatim}
  ./run_premake.bat
\end{verbatim}

Then change into the directory \path{build/windows/vs20xx}, open the
project file with the corresponding version of Visual C++ and build
the project.

The compiled binaries will end up in the directory \path{bin}.

\input{include/gpl_v3.tex}

\end{document}


%%% Local Variables:
%%% mode: latex
%%% mode: outline-minor
%%% TeX-command-default: "Rubber"
%%% TeX-PDF-mode: t
%%% ispell-local-dictionary: "british"
%%% End:
