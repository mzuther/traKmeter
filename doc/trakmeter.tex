\input{include/settings.sty}
\input{include/hyphenation.sty}

\title{traKmeter}
\author{Martin Zuther}

\begin{document}

\title{traKmeter}

\subtitle{
  \normalsize{\textrm{\textmd{
        \vfill
        Loudness meter for correctly setting up \\
        tracking and mixing levels
        \vfill
        \vspace{1.5em}
        \includegraphics[scale=0.375,clip]{include/images/trakmeter.png}
        \vfill
      }}}
}

\author{}

\date{\emph{Last edited on \today}}

\dedication{
  \includegraphics[scale=0.65,clip]{include/images/cc-by-sa.png}
  \vspace{0.25em}

  This documentation by \href{http://www.mzuther.de/}{Martin Zuther}
  is licensed under a
  \href{http://creativecommons.org/licenses/by-sa/4.0/}{Creative
    Commons Attribution-ShareAlike 4.0 International License} with the
  exception of trademark logos.

  \vspace{2.5em}

  \includegraphics[scale=0.55,clip]{include/images/VST_Compatible_Logo_Steinberg_negative.png}

  VST is a trademark of Steinberg Media Technologies GmbH,
  registered in Europe and other countries.
}

\maketitle

\tableofcontents

\clearpage  % layout

\chapter{Digital recordings}
\label{chap:digital_recordings}

The digital revolution brought many advantages to the field of audio
processing: higher fidelity, less noise, non-degrading copies and the
endless possibilities of digital signal processing.  Unfortunately,
however, digital audio also introduced some problems of its own.

Whereas the analogue domain is relatively inert against very high
levels (overdriving some analogue equipment actually sounds pretty
good), the digital domain punishes even small transgressions into
forbidden territory with harsh clipping.

And while digital audio can be transferred without loss in quality, it
is degraded by each and every calculation, be it a simple change in
level, equalisation or a fancy effect.  Crossing domains from analogue
to digital and \emph{vice versa} leads to additional degradation.
Finally, changes in bit depth or sampling rate, jitter and
inter-sample peaks are nothing for the weak of heart.

However, most of these obstacles can easily be overcome by proper gain
staging, dithering, minimising the crossing of domains and choosing
appropriate bit depths and sampling rates right from the beginning.

If you carefully choose, test and operate your equipment, you're well
on your way to pure audio bliss \dots

\section{Definitions}
\label{sec:definitions}

\begin{tabular}{p{0.375 \textwidth}p{0.525 \textwidth}}

  \textbf{RMS} \newline
  \emph{root mean square} &
  statistical method used for calculating an average from fluctuating
  values \\[0.5em]

  \textbf{\si{\dBu}} \newline
  \emph{decibel unterminated} &
  level ratio with an analogue reference level of \SI{0.7746}{\VRMS}
  \\[0.5em]

  \textbf{\si{\dBFS}} \newline
  \emph{decibel relative to \newline digital full-scale} &
  level ratio with a reference level equal to the maximum
  representable value of a digital signal \\[0.5em]

  \textbf{\si{\dBr}} \newline
  \emph{decibel relative to \newline reference level} &
  level ratio with an arbitrary reference level that must be specified;
  for instance, \SI{0}{\dBr} may be equal to \SI{-20}{\dBFS} \\[0.25em]

\end{tabular}

\section{Gain staging}
\label{sec:gain_staging}

The process of setting audio devices to run at optimal input and
output levels is called \emph{gain staging}.

Professional analogue audio equipment is designed to be run at a
nominal level of \SI[retain-explicit-plus]{+4}{\dBu}.  This leaves a
headroom for peaks of at least \textbf{\SI{20}{\dB}} to the clipping
point.  Thus, driving all analogue audio equipment at
\SI[retain-explicit-plus]{+4}{\dBu} ensures an optimal signal-to-noise
ratio while preventing clipping and keeping all transients intact.

Now let's transfer this to the digital domain.  First, choose an
analogue reference level for your converter.\footnote{for more
  information, see
  \href{https://www.digido.com/portfolio-item/level-practices-part-1/}{Level
    Practices (Part 1)} by Bob Katz; a reference level of
  \SI[retain-explicit-plus]{+4}{\dBu} = \SI{-15}{\dBFS} works well for
  me.}  Then, record with an average input level of
\textbf{\SI{-20}{\dBFS} RMS}.  Again, this ensures a good
signal-to-noise ratio while preventing clipping.  In addition, this
level drives your outboard equipment and most of your plug-ins at
their ``sweet spot''.

I also recommend recording with a maximum level of
\textbf{\SI{-10}{\dBFS} peak}.  This will leave enough space for
sudden jumps in level and may also improve the sound of your
recordings.  Some analogue-to-digital converters already degrade audio
when fed with levels close to digital full-scale (\SI{0}{\dBFS}),
resulting in the ``harshness'' that is often attributed to digital
audio.

\emph{Try this: record a clearly defined signal (such as a sine wave)
  and increase its level while analysing the obtained signal with a
  spectrum meter.  Overtones will probably appear at levels way below
  clipping point (see \ref{sec:converter_clipping}).  Their individual
  level may be low but will accumulate when you mix recordings -- and
  our ears are sensitive for distortion.}

Gain staging doesn't stop here.  Set up your mixer so that channel
levels are lower than subgroup levels, which in turn should be lower
than the master output levels.  No clipping should occur anywhere in
the mixer, which is especially important if you want to insert
external analogue gear.\footnote{for mixer inserts, I recommend
  maximum input and output levels of \textbf{\SI{-10}{\dBFS} peak} for
  the same reasons mentioned above} Do not overload effects or
plug-ins and (if possible) try to match their input and output
volume.\footnote{many digital signal processors use floating-point
  calculations and handle clipping gracefully; others, especially
  those modelling old analogue equipment, may clip badly}

Speaking of me, once I transitioned to proper gain staging, my
recordings and mixes became much cleaner.  And the effort involved was
pretty small!

\section{Digital audio myths}
\label{sec:digital_audio_myths}

I can almost hear you: digital recordings should be made at peak
levels close to but not exceeding \SI{0}{\dBFS}.  Unfortunately, this
misinformation has ended up in many manuals for professional audio
equipment.  But that doesn't make it any truer \dots

Even if you record at \emph{peak} levels of \SI{-24}{\dBFS}, a bit
depth of \SI{16}{\bits} would still leave you with a signal-to-noise
ratio of \SI{72}{\dB}.  That is about what you can expect from some of
the best professional analogue tape machines and recording desks!

Now think of the \SI{20}{\bits}\footnote{you only record at
  \SI{24}{\bits} because that's easier for your computer (and
  manufacturers love big numbers in their specs) -- real-world
  circuits are contaminated with thermal noise which limits the usable
  bit depth of converters; see
  \href{http://www.sengpielaudio.com/calculator-noise.htm}{here} for
  more information} you can capture with modern converters.  Recording
at peak levels of \SI{-24}{\dBFS} is \emph{fully equivalent} to using
a bit depth of \SI{16}{\bits} and recording at levels close to
clipping point.

These days, there really is no point in recording at extreme levels.
You don't have to take my word, though -- read these three Gearslutz
posts written by a renowned engineer:
\href{https://www.gearslutz.com/board/showpost.php?p=10739624&postcount=9}{\#1},
\href{https://www.gearslutz.com/board/showpost.php?p=5063154&postcount=219}{\#2}
and
\href{https://www.gearslutz.com/board/showpost.php?p=9909382&postcount=96}{\#3}.
The highly regarded manufacturer \emph{Harrison Audio} recommends a
maximum peak recording level of
\href{http://www.harrisonconsoles.com/mixbus/mixbus32c-6-live-manual/1/en/topic/gain-staging}{\SI{-15}{\dBFS}}
for their \emph{Mixbus 32C} digital audio workstation.  Finally, I
highly recommend reading
\href{https://www.soundonsound.com/techniques/gain-staging-your-daw-software}{this
  article} in \emph{Sound On Sound}.

\section{Introducing traKmeter}
\label{sec:introducing_trakmeter}

Sadly, most digital audio equipment only has peak meters.  This is
readily understandable as you want to avoid digital clippings by all
means.  However, badly chosen meter ranges and scales often render
these meters useless.  And the lack of average meters does not exactly
facilitate gain staging.

When I had realised this, I started coding traKmeter.  It has evolved
with my growing knowledge and recording experience, but the underlying
ideas haven't changed at all.

\section{Converter clipping}
\label{sec:converter_clipping}

This experiment was conducted with a \emph{RME Fireface 800} interface
(\SI{96}{\kilo\hertz}, 24 bits, \SI{0}{\dBFS} set to
\SI[retain-explicit-plus]{+19}{\dBu}).  Signals were sent to an output
connected directly to an input (\textbf{red lines}).  A floating input
served as reference (\textbf{blue lines}).

\subsection{Single sine wave}

\includegraphics[scale=0.19,clip]{include/images/clipping_sine_ff800.png}

\subsection{Complex signal (several sine waves)}

\includegraphics[scale=0.19,clip]{include/images/clipping_harmonics_ff800.png}

\chapter{traKmeter}
\label{chap:trakmeter}

\begin{wrapfigure}{r}{0.21\linewidth}
  \includegraphics[scale=\screenshotscale,clip]{include/images/level_meter_complete.png}
\end{wrapfigure}

traKmeter consists of two meters, a peak meter on top and an average
meter below.  The meters are separated by an orange signal LED and
consist of a green area that is enclosed by a blue one (lower levels)
or a yellow and red one (higher levels).

The average meter's green area is centred around the
\textbf{\SI{-10}{\dBr} RMS} mark.  With a reference level of
\SI{-10}{\dBFS}, this corresponds to the \textbf{\SI{-20}{\dBFS} RMS}
we have determined to be the optimal average recording level in the
digital domain.

The peak meter's yellow area reaches up to \textbf{\SI{0}{\dBr} peak}.
Given the same reference level, this corresponds to
\textbf{\SI{-10}{\dBFS} peak} and peak levels shouldn't exceed this
level.

Thus, by keeping the meter's readout \textbf{in the green} and yellow
areas and \textbf{out of the red} areas, you will automatically track
at optimal audio levels.  It's as simple as that!

\section{Tracking with traKmeter}
\label{sec:tracking_with_trakmeter}

Open up an instance of traKmeter and set it up so that it measures
your audio input.  That can be done either by starting the stand-alone
version and connecting it to one or more input channels of your
analogue-to-digital converter, or by inserting a plug-in instance into
an input channel of your digital audio workstation (no latency is
added).

In the second case, take care that your digital audio workstation
doesn't add additional headroom and that no processing takes place
before traKmeter.  This can be ascertained by feeding calibration
tones into your converter or by directly comparing the readouts of
stand-alone and plug-in version.

\begin{wrapfigure}{r}{0.42\linewidth}
  \includegraphics[scale=0.425,clip]{include/images/trakmeter_optimal.png}
\end{wrapfigure}

Now, feed the signal you want to record into an audio input channel
and adjust its level (in the analogue domain!).  Try to set the input
level so that it falls into the average meter's
\textbf{\SI{-10}{\dBr}} area.

Make sure that peak levels very rarely (if ever) exceed
\textbf{\SI{0}{\dBr}}.  In case both conditions cannot be met
simultaneously, adjust the peak level only.

\section{Mixing with traKmeter}
\label{sec:mixing_with_trakmeter}

When you get someone else's tracks for mixing, chances are that they
have been recorded far too hot.  While you can't change that, you
should adjust the tracks to optimal loudness in the gain staging
phase.  This is easily accomplished using traKmeter and either the
mixer's trim knob or a (properly dithered!) gain plug-in.  Just make
sure that the gain change precedes all future processing.

Mixing levels will now be much lower than what you might be used to.
This can easily be corrected by either adjusting the output gain of
your subgroups or by inserting a gain plug-in in your master track.

To preserve all transients, the final loudness of your mix should stay
within certain average level ranges.  My plug-in
\href{http://code.mzuther.de/}{\textbf{K-Meter}} may help you with
setting up correct mixing levels.  Remember that smashed transients
will be gone forever, whereas you can always bring up the volume
during mastering!

\chapter{Installation}
\label{chap:installation}

In order to use the pre-compiled binaries, simply extract the
traKmeter files from the downloaded archive.  For the plug-ins, you'll
then have to move the extracted files to your respective plug-in
folder.

traKmeter requires a processor which supports the SSE2 instruction
set.  On Windows, you might also have to install the
\href{https://www.visualstudio.com/downloads/}{Visual C++
  Redistributable for Visual Studio 2017}.

Should the stand-alone version ever fail to start, you can reset its
settings by deleting \path{traKmeter (Stereo).settings} or
\path{traKmeter (Multi).settings}.  These files are located in
\path{~/.config} (GNU/Linux) or \path{%appdata%\.config\} (Windows).

\chapter{Controls}
\label{chap:controls}

\section{Reference level buttons}
\label{sec:reference_level_buttons}

\begin{wrapfigure}{r}{0.14\linewidth}
  \includegraphics[scale=\screenshotscale,clip]{include/images/button_recording_level.png}
\end{wrapfigure}

Set your reference level (corresponding to the preferred maximum peak
recording level) with these buttons.  To follow the advice in
\ref{sec:gain_staging}, leave this at the default setting of
\textbf{\SI{-10}{\dBFS}}.

If you record very dynamic material or have problems getting a clean
signal, you can try the other settings.  They shift both meter scales
down by \SIrange{5}{10}{\dB} and add additional headroom at the cost
of a higher noise floor.  Which you probably won't even notice, as the
signal-to-noise ratio of modern converters is extremely high.

\section{Reset button}
\label{sec:reset_button}

\begin{wrapfigure}{r}{0.14\linewidth}
  \includegraphics[scale=\screenshotscale,clip]{include/images/button_reset_on.png}
  \newline \vspace{-0.9\baselineskip}
  \includegraphics[scale=\screenshotscale,clip]{include/images/button_reset_off.png}
\end{wrapfigure}

Click on this button to reset all meters.  This action will also
reload the current skin and re-draw everything.

\section{Select a skin}

\begin{wrapfigure}{r}{0.14\linewidth}
  \includegraphics[scale=\screenshotscale,clip]{include/images/button_skin_on.png}
  \newline \vspace{-0.9\baselineskip}
  \includegraphics[scale=\screenshotscale,clip]{include/images/button_skin_off.png}
\end{wrapfigure}

Click on this button to select the currently used traKmeter skin.  You
can also set a default skin that will be loaded when new plug-in
instances are started.

\section{Validation button}
\label{sec:validation_button}

\begin{wrapfigure}{r}{0.14\linewidth}
  \includegraphics[scale=\screenshotscale,clip]{include/images/button_validate_on.png}
  \newline \vspace{-0.9\baselineskip}
  \includegraphics[scale=\screenshotscale,clip]{include/images/button_validate_off.png}
\end{wrapfigure}

Click on this button to open the \textbf{validation window} (see
\ref{chap:validation}) which allows you to play an audio file through
traKmeter and dump internal data.  During validation, the button will
light up and clicking on it will stop validation early.

On Linux, dumped data will be written to \path{stderr}, so just start
the traKmeter stand-alone or your plug-in host from the shell and
watch the output coming.  On Windows, you can use DebugView by
Sysinternals (stand-alone) or have a look at Ableton Live's log files
(plug-in).  If none of that works, you might have to start either the
stand-alone or your plug-in host from a debugger.

As a side note, \textbf{SMA(50)} designates the simple moving average
of \num{50} values, a neat way to emphasise trends and eliminate
short-term fluctuations.

\section{About button}

\begin{wrapfigure}{r}{0.14\linewidth}
  \includegraphics[scale=\screenshotscale,clip]{include/images/button_about_on.png}
  \newline \vspace{-0.9\baselineskip}
  \includegraphics[scale=\screenshotscale,clip]{include/images/button_about_off.png}
\end{wrapfigure}

Clicking on this button will open the \textbf{about window} where you
will be informed about version number, contributors, copyright and the
GNU General Public License.

\section{Display license}

\begin{wrapfigure}{r}{0.15\linewidth}
  \includegraphics[scale=\screenshotscale,clip]{include/images/button_gpl_on.png}
  \newline \vspace{-0.9\baselineskip}
  \includegraphics[scale=\screenshotscale,clip]{include/images/button_gpl_off.png}
\end{wrapfigure}

This button is located in the \textbf{about window} and does not only
advertise that you are using free software licensed under the
\textbf{GNU General Public License} -- when clicked, it will also open
the license's website in your web browser \dots

\chapter{Meters}
\label{chap:meters}

All meters possess a flat frequency response.  Meter scales are in
\si{\dBr} and their reference level is adjustable (see
\ref{sec:reference_level_buttons}).  This way, scales remain the same
even when the reference level changes.

\section{Peak level meter}

\begin{wrapfigure}[6]{r}{0.21\linewidth} %% layout [number of narrow lines]
  \includegraphics[scale=\screenshotscale,clip]{include/images/level_meter_peak.png}
\end{wrapfigure}

This meter shows the current peak level in \si{\dBr}.  Rise time is
one sample and fall time is \SI{8.67}{\dB\per\second}.  Peak levels
exceeding \SI{0}{\dBr} are displayed on the red LED marked ``OVR''.

The highest encountered peak level will be held indefinitely until the
meter is reset.

\newpage %% layout

\section{Average level meter}

\begin{wrapfigure}{r}{0.21\linewidth}
  \includegraphics[scale=\screenshotscale,clip]{include/images/level_meter_average.png}
\end{wrapfigure}

The average level meter uses an averaging period of \num{1024}
samples.  It has been calibrated so that sine wave signals read the
same on both peak and average meters.  Similar to a VU meter, it takes
\SI{300}{\milli\second} for the meter to reach \SI{99}{\percent} of
the final reading.  There is no overshoot, however.

Peaks will be held for \SI{10}{\second} and then fall with a speed of
\SI{8.67}{\dB\per\second}.

\section{Signal meter}

\begin{wrapfigure}{r}{0.21\linewidth}
  \includegraphics[scale=\screenshotscale,clip]{include/images/level_meter_signal.png}
\end{wrapfigure}

The orange signal meter detects peak levels at a threshold of
\SI{-60}{\dBFS}.  It has a rise time of one sample and fades out when
the level falls below the threshold.

\chapter{Recording advice}
\label{chap:recording_advice}

Over the years, I have learned how to and how not to record, and this
seems like a good place to summarise my knowledge.  For controversial
advice, I will try to reference the opinion of professionals.

\begin{description}

\item[Use a good pre-amplifier.]  ``Good'' doesn't mean your
  pre-amplifier has to have a lot of channels or features.  To the
  contrary!  Go for a simple design and invest your money in
  professional quality instead.  Recordings made with a good
  pre-amplifier sound better and make mixing much easier -- the tracks
  simply seem to fall into place.

\item[ Use the pre-amplifier's gain control.]  If necessary, crank up
  the gain control to yield the correct output level.  Do not fear the
  pre-amplifier's internal noise -- boosting a low-gain recording in
  later stages will likely result in even more noise.

\item[Avoid unbalanced equipment.]  Run all signals on balanced lines
  with a nominal level of \SI[retain-explicit-plus]{+4}{\dBu}.  If you
  can't, use DI boxes and transformers to convert the signals as early
  in the audio chain as possible.

\item[Use short audio chains.]  All equipment adds noise or may
  otherwise degrade audio, so keep your audio recording chains short
  and simple.  This has the additional benefit that you can focus on
  recording.

  Here is an example: instead of routing your mixer between
  pre-amplifier and converter, connect the mixer to your converter's
  outputs.  This simple change can lead to much better recordings
  (especially with cheap mixers) and the artist will still be able to
  hear playback and herself during recording.

\item[Work at a defined reference level.]  See the article
  \href{https://www.digido.com/portfolio-item/level-practices-part-1/}{Level
    Practices (Part 1)} by Bob Katz.

\item[Record at lower levels.]  During recording, do not let peak
  levels exceed \textbf{\SI{-10}{\dBFS}}.  For an in-depth explanation
  and references, see \ref{sec:gain_staging}.

\item[Record in mono.]  Most audio sources do not contain stereo
  information that is useful in a mixing context (notable exceptions
  are audience recordings, orchestras and occasionally pianos).  The
  pseudo-stereo effects of synthesisers may even cause phasing issues.

  Recording such sources in stereo will only waste space on your hard
  disk.  So don't!

\item[Use high bit depths.]  Do yourself a favour and record at a bit
  depth of \SI{24}{\bits} instead of \SI{16}{\bits}.  The additional
  bits allow recording at lower levels and provide an incredible
  amount of extra detail.  When disk space runs low, choose more bits
  over a higher sampling rate.  See the
  \href{http://www.aes.org/par/d/\#data_converter_bits}{AES Pro Audio
    Reference} for more information.

  The usable bit depth of converters is limited to approximately 20
  bits by thermal noise.  However, the mixer in your digital audio
  workstation should use floating point numbers with at least
  \SI{32}{\bits}.  Calculation errors are inevitable in digital signal
  processing, and more bits mean smaller (and thus quieter) errors.

\item[Use sensible sampling rates.]  According to a
  \href{http://www.lavryengineering.com/pdfs/lavry-white-paper-the_optimal_sample_rate_for_quality_audio.pdf}{white
    paper} by Dan Lavry, manufacturer of high-quality converters,
  \SI{88.1}{\kilo\hertz} and \SI{96}{\kilo\hertz} are the preferred
  rates for recording audio.  As support for \SI{88.1}{\kilo\hertz} is
  limited, I prefer to work at \SI{96}{\kilo\hertz}.

\item[Concentrate on recording.]  When tracking, try to not interfere
  with the flow of the session.  Just keep editing and mixing to the
  bare minimum.

\item[Fix it now.]  Contrary to popular belief, you cannot ``fix it in
  the mix'' -- a bad recording is nothing more than a bad recording.
  So editing and tools like Auto-Tune and extreme EQ should be seen as
  a last resort.  Apart from the precious time lost in editing, it's
  easy to kill all of a track's vibe in the process.

  Instead, keep recording until you capture a great take.  Treat your
  room acoustically and in terms of positive vibe.  Experiment with
  microphone placement.  And try absolutely everything that may help
  the artist in achieving a stunning performance!

\item[Make it exciting.]  A lot of today's music sounds like (and
  actually is) one short loop that was ``arranged'' by muting
  different tracks at different times.  This takes away all the small
  inaccuracies of human players and often leads to boring and lifeless
  songs.

  So think of a good arrangement before you even start recording.
  Instead of looping a track, record a couple of takes and comp the
  best ones.  You'll be surprised at the difference it makes!

\end{description}

\chapter{Validation}
\label{chap:validation}

I have gone to great lengths to ensure that the meters read correctly.
You want to validate for yourself?  Just download and extract the
source code.  The directory \path{validation} contains instructions
and FLAC-compressed wave files.  A word of warning: these audio files
may \textbf{damage your ears} and speakers, so please watch your
monitor levels!

Begin by starting traKmeter.  If in a Bash shell, try this:

\begin{VerbatimBoth}
  ./trakmeter_stereo 2>&1 | tee /tmp/validate.log
\end{VerbatimBoth}

\begin{wrapfigure}{r}{0.32\linewidth}
  \includegraphics[scale=0.60,clip]{include/images/dialog_validation.png}
\end{wrapfigure}

After opening the \textbf{validation window} (see
\ref{sec:validation_button}), click on the ellipsis button (the one
with the dots) to select an audio file for playback through traKmeter.
Please make sure that the sampling rates of your host (\textbf{Host
  SR}) and the audio file match, otherwise the results will not be
correct.

Now, select which \textbf{variables} (if any) should be dumped.  You
may also restrict dumped data to a specific audio \textbf{channel}.
Check \textbf{CSV} if you want to feed the output to a parser.

Finally, click on the \textbf{validate} button to reset all meters and
start playback of the selected audio file.  All audio input will be
discarded during playback and for an additional twenty seconds.  To
stop playback early, simply click on the \textbf{validate} button
again.

\section{Validation status}

\begin{minipage}{1.0\linewidth}
  \renewcommand{\thempfootnote}{\arabic{mpfootnote}}
  \begin{tabular}{>{\bfseries}llcc}

    &
    \textbf{Test} &
    \textbf{Valid} \\

    Average level meter &
    visuals &
    \Checkmark{} \\

    &
    readout &
    \Checkmark{} \\

    Peak level meter &
    visuals &
    \Checkmark{} \\

    &
    readout &
    \Checkmark{} \\

    Signal meter &
    visuals &
    \Checkmark{} \\

  \end{tabular}
\end{minipage}

\chapter{Help needed}
\label{chap:help_needed}

As traKmeter was coded using cross-platform code, it should be easy to
compile on Mac OS X.  Unfortunately, I happen to not have a Mac \dots

In case you want to help, please see the next chapter for an email
address.  You’ll need sufficient experience in coding, compiling and
debugging, though, so no beginners please!

\chapter{Final words}
\label{chap:final_words}

I want to thank \textbf{Rickard} of Interfearing Sounds for asking me
how to use K-Meter for tracking.  This question and the following
thoughts really got traKmeter started.  I'd like to thank
\textbf{bram@smartelectronix} for his code to calculate logarithmic
rise and fall times.  I must also thank the \textbf{beta testers} and
\textbf{users of traKmeter} for sending kind words, suggestions and
bug reports.  Finally, I want to thank the \textbf{open source
  community} for making all of this possible.

Although coding traKmeter has been a lot of fun, it has also been a
lot of work.  So if you like traKmeter, why not
\href{http://www.mzuther.de/}{send me an email} and tell me so?  Write
a few words about yourself, send suggestions for future updates or
volunteer to create a nice skin.  I also really enjoy listening to
music that you have produced using my software \dots

\emph{Thanks for using free software.  I hope you'll enjoy it!}

\appendix

\chapter{How to build traKmeter}
\label{chap:build_trakmeter}

\section{Dependencies}
\label{sec:dependencies}

\subsection{premake}
\label{sec:dependencies_premake}

\begin{tabbing}
  \hspace*{6em}\=\=\kill

  Importance:  \> required \\
  Version:     \> 5.0.0 (alpha14) \\
  License:     \> BSD \\
  Homepage:    \> \href{https://premake.github.io/}{premake.github.io}
\end{tabbing}

\subsubsection{Installation}

Place the binary somewhere in your \path{PATH}.  Depending on your
platform, you should run \path{premake} using the scripts
\path{Builds/run_premake.sh} or \path{Builds/run_premake.bat}.

To change the premake file using Jinja templates, you'll also have to
install the necessary dependencies.

\subsection{Compilers}

\begin{tabbing}
  \hspace*{6em}\=\=\kill

  Importance:  \> required \\
  Linux:       \> Clang 6.0 (or gcc 7.5.0) \\
  Windows:     \> Visual Studio 2017 \\
  License:     \> proprietary (Visual Studio) / Open Source \\
\end{tabbing}

Use premake (\ref{sec:dependencies_premake}) to generate the Make
files (or project) files needed by different compilers.

\emph{Different compiler versions may work, and premake supports other
  compiler tool sets as well.  But in this case, you're on your own!}

\subsection{JUCE library}

\begin{tabbing}
  \hspace*{6em}\=\=\kill

  Importance:  \> required \\
  Version:     \> 5.4.7 \\
  License:     \> ISC and GPL v3 (among others) \\
  Homepage:    \> \href{http://www.juce.com/}{www.juce.com}
\end{tabbing}

\subsubsection{Installation}

Extract the archive into the directory \path{libraries/juce}.

\subsection{Virtual Studio Technology SDK}

\begin{tabbing}
  \hspace*{6em}\=\=\kill

  Importance:  \> optional \\
  Version:     \> 2.4 / 3.6.14 \\
  License:     \> proprietary / GPL v3 \\
  Homepage:    \> \href{http://www.steinberg.net/en/company/developer.html}{www.steinberg.net}
\end{tabbing}

\subsubsection{Installation}

Extract the archives into the directories \path{libraries/vst2} and
\path{libraries/vst3}.  The proprietary VST2 SDK is not available
anymore.  \textbf{You may only distribute VST2 plug-ins if you have
  signed the old license agreement!}

\subsection{Python}

\begin{tabbing}
  \hspace*{6em}\=\=\kill

  Importance:  \> optional \\
  Version:     \> 3.6 (or higher) \\
  License:     \> Python Software Foundation License \\
  Homepage:    \> \href{http://www.python.org/}{www.python.org}
\end{tabbing}

You'll only need Python if you want to auto-generate files from Jinja
templates.

\subsubsection{Installation (Windows)}

You can download an installer from the website.

\subsection{Jinja}

\begin{tabbing}
  \hspace*{6em}\=\=\kill

  Importance:  \> optional \\
  Version:     \> 2.10 (or higher) \\
  License:     \> BSD \\
  Homepage:    \> \href{http://jinja.pocoo.org/}{jinja.pocoo.org}
\end{tabbing}

You'll only need Jinja if you want to auto-generate files such as the
premake file from templates (see \ref{sec:dependencies_premake}).

\subsection{Artistic Style}

\begin{tabbing}
  \hspace*{6em}\=\=\kill

  Importance:  \> optional \\
  Version:     \> 3.1 \\
  License:     \> LGPL v3 \\
  Homepage:    \> \href{http://astyle.sourceforge.net/}{astyle.sourceforge.net}
\end{tabbing}

This application formats the code so it looks more beautiful and
consistent.  Thus, you only have to install it if you plan to help me
with coding traKmeter.

\subsubsection{Installation}

Place the binary somewhere in your \path{PATH}.  Depending on your
platform, you should run \path{astyle} using the scripts
\path{Source/format_code.sh} or \path{Source/format_code.bat}.

\newpage %% layout

\section{GNU/Linux}

\subsection{Environment}

To build traKmeter yourself, I recommend setting up a \texttt{chroot}
environment.  This is fast and easy to do on Debian-based systems and
might save you a \textbf{lot} of trouble.  At the time of writing, I'm
using Linux Mint 19, but the procedure should be similar on your
distribution of choice.

Start by installing the necessary packages:

\begin{VerbatimBoth}
  sudo apt-get install debootstrap schroot
\end{VerbatimBoth}

Then install the \texttt{chroot} base system by executing the
following statements:

\begin{Verbatim32}
  sudo debootstrap --variant=buildd \
    --arch i386 bionic \
    /srv/chroot/bionic_i386 \
    http://archive.ubuntu.com/ubuntu
\end{Verbatim32}

\begin{Verbatim64}
  sudo debootstrap --variant=buildd \
    --arch amd64 bionic \
    /srv/chroot/bionic_amd64 \
    http://archive.ubuntu.com/ubuntu
\end{Verbatim64}

Running \path{debootstrap} will take some time.  Meanwhile, add the
following lines to \path{/etc/schroot/schroot.conf} (make sure you
remove all preceding white space so that each line begins in the first
column):

\begin{VerbatimBoth}
  [bionic-i386]
  description=Ubuntu bionic (i386)
  directory=/srv/chroot/bionic_i386
  profile=default
  personality=linux32
  type=directory
  users=username

  [bionic-amd64]
  description=Ubuntu bionic (amd64)
  directory=/srv/chroot/bionic_amd64
  profile=default
  personality=linux
  type=directory
  users=username
\end{VerbatimBoth}

Please make the necessary changes to \texttt{username}.  If you
experience problems, you can try to change \texttt{bionic} to a
release name such as \texttt{wheezy}.

When \path{debootstrap} is done, log in as superuser:

\begin{Verbatim32}
  sudo schroot -c bionic-i386
\end{Verbatim32}

\begin{Verbatim64}
  sudo schroot -c bionic-amd64
\end{Verbatim64}

You'll have to change the file \path{/etc/apt/sources.list} first
(ignore the line break, it should be a single line):

\begin{VerbatimBoth}
  deb http://archive.ubuntu.com/ubuntu bionic
  main restricted universe
\end{VerbatimBoth}

Now install a few packages -- \path{less} and \path{vim} are optional,
but might come in handy:

\begin{VerbatimBoth}
  apt-get update
  apt-get -y install bash-completion clang \
    libasound2-dev libjack-jackd2-dev \
    mesa-common-dev xorg-dev less vim
  apt-get clean
\end{VerbatimBoth}

If you like \path{bash} completion, you might also want to open the
file \path{/etc/bash.bashrc} and unquote these lines:

\begin{VerbatimBoth}
  # enable bash completion in interactive shells
  if [...]
    [a couple of lines...]
  fi
\end{VerbatimBoth}

Finally, log out and log in as normal user:

\begin{Verbatim32}
  schroot -c bionic-i386
\end{Verbatim32}

\begin{Verbatim64}
  schroot -c bionic-amd64
\end{Verbatim64}

In this \path{chroot} shell, install the dependencies
(\ref{sec:dependencies}).  Congratulations -- you are now ready to
build traKmeter!

\subsection{Build}

After preparing the dependencies, start your \texttt{chroot}
environment

\begin{Verbatim32}
  schroot -c bionic-i386
\end{Verbatim32}

\begin{Verbatim64}
  schroot -c bionic-amd64
\end{Verbatim64}

change into the directory \path{build} and execute

\begin{VerbatimBoth}
  ./run_premake.sh
  make config=CFG TARGET
\end{VerbatimBoth}

where \application{CFG} is one of \application{debug\_x32},
\application{debug\_x64}, \application{release\_x32} and
\application{release\_x64}, and \application{TARGET} is the version
you want to compile, such as \application{linux\_standalone\_stereo}.

In case you run into problems, you can try to switch compilers by
opening the file \texttt{run\_premake.sh} and using the premake
options \texttt{--cc=clang} or \texttt{--cc=gcc}.

The compiled binaries will end up in the directory \path{bin}.

\section{Microsoft Windows}

\subsection{Build}

After preparing the dependencies, change into the directory
\path{build} and execute

\begin{VerbatimBoth}
  ./run_premake.bat
\end{VerbatimBoth}

Then change into the directory \path{Builds/windows/vs20xx}, open the
project file with the corresponding version of Visual C++ and build
the project.

The compiled binaries will end up in the directory \path{bin}.

\chapter{Licenses}

\scriptsize
\input{include/gpl_v3.tex}
\normalsize

\scriptsize
\input{include/cc-by-sa-4.0.tex}
\normalsize

\end{document}


%%% Local Variables:
%%% mode: latex
%%% mode: outline-minor
%%% TeX-command-default: "Rubber"
%%% TeX-master: t
%%% TeX-PDF-mode: t
%%% ispell-local-dictionary: "british"
%%% End:
