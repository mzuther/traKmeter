\input{include/settings.sty}
\input{include/hyphenation.sty}

\title{traKmeter}
\author{Martin Zuther}

\begin{document}

\title{traKmeter}

\subtitle{
  \normalsize{\textrm{\textmd{
        \vfill
        Loudness meter for correctly setting up \\
        tracking and mixing levels
        \vfill
        \vspace{1.5em}
        \includegraphics[scale=0.45,clip]{include/images/trakmeter.png}
        \vfill
      }}}
}

\author{}

\date{\emph{Last edited on \today}}

\dedication{
  \includegraphics[scale=0.65,clip]{include/images/cc-by-sa.png}
  \vspace{0.25em}

  This documentation by \href{http://www.mzuther.de/}{Martin Zuther}
  is licensed under a
  \href{http://creativecommons.org/licenses/by-sa/4.0/}{Creative
    Commons Attribution-ShareAlike 4.0 International License} with the
  exception of the GPL, VST and ASIO logos.
}

\maketitle

\tableofcontents

\clearpage  % layout

\chapter{Digital recordings}
\label{chap:digital_recordings}

The digital revolution brought a lot of advantages to the field of
audio processing such as higher fidelity, less noise and non-degrading
copies.  Unfortunately, however, digital audio also introduced some
problems of its own.

Whereas the analog domain is relatively inert against very high levels
(overdriving some analog equipment actually sounds pretty good), the
digital domain punishes even small transgressions into forbidden
territory with harsh clipping.

And while digital audio can be transferred without loss in quality, it
is degraded by each and every calculation, be it a simple change in
level, equalisation or a fancy effect.  Crossing domains from analog
to digital and \emph{vice versa} leads to additional degradation.
Finally, changes in bit depth and sample rate, jitter and inter-sample
peaks are nothing for the weak of heart.

However, most of these obstacles can be overcome easily by proper gain
staging, minimising the crossing of domains and choosing appropriate
bit depths and sample rates.  If you also learn how to properly test
and operate your equipment, you're well on your way to pure audio
bliss \dots

\section{Gain staging}
\label{sec:gain_staging}

Professional analog audio equipment is designed to be run at a nominal
level of \textbf{\SI[addsign=all]{+4}{\dBu}} (\SI{1.23}{\VRMS}) and
leaves a headroom for peaks of about \SI{20}{\dB}.  This in turn is
consistent with the maximum crest factor of analog audio signals.

Thus, driving all analog audio equipment at \SI[addsign=all]{+4}{\dBu}
ensures an optimal signal-to-noise ratio while preventing clipping and
keeping all transients intact.  The process of setting audio devices
to run at optimal input and output levels is called \emph{gain
  staging}.

Now let's transfer this to the digital domain.  As the maximum crest
factor of analog audio signals amounts to \SI{20}{\dB}, we'll adjust
the headroom accordingly by setting our average input and output
levels to \textbf{\SI{-20}{\dBFS} RMS}.

Again, this ensures a good signal-to-noise ratio while preventing
clipping.  Maybe even more important, this level also drives (most of)
your digital audio equipment and plug-ins at their respective ``sweet
spot''.

Another recommendation is that peak levels should not exceed
\textbf{\SI{-9}{\dBFS}}
(\href{http://tech.ebu.ch/publications/r068}{EBU R68-2000}) during
tracking.  This will leave enough space for sudden jumps in level and
also for inter-sample peaks, audio peaks that lie \emph{in between}
two successive samples and may lead to unpredictable clipping during
digital-to-analog conversion.

Some analog-to-digital converters also degrade audio when fed with
input levels close to digital full-scale (\SI{0}{\dBFS}), resulting in
the ``harshness'' often attributed to digital audio -- my first sound
card definitely suffered from this.  So lowering your input levels as
described above may also improve your overall sound.

Finally, we'll emphasise the newly designated headroom and shift the
meter scales by \SI[addsign=all]{+20}{\dB}.  Thus, the optimal average
audio level is designated \textbf{\SI{0}{\dB} RMS}, while the maximum
peak audio level becomes \textbf{\SI[addsign=all]{+11}{\dB}}.  As a
nice side effect, our new scale corresponds to Bob Katz's
\textbf{\href{http://www.digido.com/level-practices-part-2-includes-the-k-system.html}{K-20
    scale}}.

\section{Digital audio myths}
\label{sec:digital_audio_myths}

I can almost hear you: you have heard that digital recordings should
be performed at peak levels close to but not exceeding \SI{0}{\dBFS}
(digital full-scale).  Heck, this misinformation has ended up in the
manuals of some professional audio equipment.  But for the reasons
given above it is plain wrong.

Let's look at a worst-case scenario: even if your recordings
\emph{peak} at \SI{-20}{\dBFS} and you discard the least significant
bit (some people claim that it mostly consists of errors), a bit depth
of 16 bit would still leave you with a signal-to-noise ratio of
\SI{70}{\dB}.  That is about what you can expect from some of the best
professional analog tape machines and recording desks -- and we're not
even talking of 24 bit.

\begin{wrapfigure}{r}{0pt}
\includegraphics[scale=0.24,clip]{include/images/level_meter_otari.png}
\end{wrapfigure}

If you don't believe me yet, take a look at my professional 16-bit
hard disk recorder (Otari PD-80): its analog inputs and outputs are
aligned to ``\SI[addsign=all]{+4}{\dBu} (\textbf{\SI{-15}{\dB}} from
digital full-scale)''.  The manufacturer has even marked this level on
the meter bridge (small triangle on the photo).  Although I admit that
the mark is only useful for audio alignment, given that it sits on a
peak meter \dots

There is also a great thread over at Gearslutz (``The Reason Most ITB
mixes don’t Sound as good as Analog mixes'') well worth reading.  Here
are links to the
\href{http://www.gearslutz.com/board/5062929-post1.html}{first post}
and two other selected posts
(\href{http://www.gearslutz.com/board/5064831-post1874.html}{\#1874}
and
\href{http://www.gearslutz.com/board/5609740-post3614.html}{\#3614}).

\section{Introducing traKmeter}
\label{sec:introducing_trakmeter}

Most digital audio equipment sadly only has peak meters.  This is
readily understandable as you want to avoid digital clippings by all
means.  However, the lack of average meters makes correct gain staging
almost impossible.

For gain staging, you need average meters or -- even better -- a
combination of peak and average meters.  And this is were traKmeter
comes in.

\chapter{traKmeter}
\label{chap:trakmeter}

\begin{wrapfigure}{r}{0.21\linewidth}
\includegraphics[scale=\screenshotscale,clip]{include/images/level_meter_complete.png}
\end{wrapfigure}

traKmeter consists of two meters, a peak meter on top and an average
meter below.  The meters are separated by an orange signal LED and
consist of an area of green LEDs that is enclosed by either blue ones
(lower levels) or yellow and then red ones (higher levels).

You may have noticed that the average meter's green area is centred
around the \textbf{\SI{0}{\dB} RMS} mark.  This number should be
vaguely familiar.  Remember, it corresponds to \SI{-20}{\dBFS} RMS,
the level we have determined to be the optimal average audio level in
the digital domain.

A fully lit yellow LED on the peak meter's top end corresponds to a
level of \textbf{\SI[addsign=all]{+11}{\dB}} (or \SI{-9}{\dBFS}).
Again, this number should be familiar: peak levels in the digital
domain shouldn't exceed this level.

Thus, by keeping the meter's readout in the green areas and from
entering the yellow and especially the red areas on top of each meter,
you will automatically track at optimal audio levels.  It's as simple
as that!

\section{Tracking with traKmeter}
\label{sec:tracking_with_trakmeter}

Open up an instance of traKmeter and set it up so that it measures
your audio input.  That can be done either by starting the standalone
version and connecting it to one or more input channels of your sound
card, or by inserting a plug-in instance into an input channel of your
digital audio workstation.

In the second case, take care that your digital audio workstation
doesn't add additional headroom and that no processing takes place
before traKmeter.  This can be ascertained by feeding calibration
tones into your sound card or by directly comparing the readouts of
standalone and plug-in version.

\begin{wrapfigure}{r}{0.45\linewidth}
\includegraphics[scale=0.45,clip]{include/images/trakmeter_optimal.png}
\end{wrapfigure}

Now, feed the signal you want to record into an audio input channel
and adjust its level (in the analog domain!) using traKmeter.  Try to
set the input level so that transients fall into the average meter's
\textbf{\SI{0}{\dB} RMS} area.  Make sure that peak levels never
exceed \textbf{\SI[addsign=all]{+11}{\dB}}.  In case both conditions
cannot be met simultaneously, adjust the peak level only.  See the
image to the right for a visual clue.

\section{Mixing with traKmeter}
\label{sec:mixing_with_trakmeter}

When you get someone else's tracks for mixing, chances are that they
have been recorded far too hot.  While you can't change that, you
might want to adjust them to optimal loudness so that your upcoming
mix is not ruined.

\emph{If the original recordings were made with poor equipment and you
  have the time, it may well be worth to \textbf{re-record} all tracks
  through a really good preamp and adjust their loudness at the same
  time.  Depending on the preamp, the results can be stunning!}

Another option is to insert traKmeter on each channel as first
plug-in, enable the ``Mixing'' button (see \ref{sec:mixing_button})
and adjust volume using the gain knob.

In any case, mixing levels will now be much lower than what you are
used to.  This can easily be corrected by either adjusting the output
gain of your subgroups or by inserting a gain plug-in in your master
track.

To preserve all transients, the final loudness of your mix should stay
within \textbf{\SI{-20}{\dBFS} RMS} and \textbf{\SI{-16}{\dBFS} RMS}
(or between \textbf{\SI{0}{\dB} RMS} and
\textbf{\SI[addsign=all]{+4}{\dB} RMS} on the K-20 scale).  Remember
that smashed transients will be gone forever, whereas you can always
bring up the volume during mastering!  My plug-in
\href{http://code.mzuther.de/}{\textbf{K-Meter}} and its K-20 scale
may help you with setting up correct mixing levels.

\chapter{Installation}
\label{chap:installation}

In order to use the pre-compiled binaries, simply extract the
traKmeter files from the downloaded archive.  For the plug-ins, you'll
then have to move the extracted files to your respective plug-in
folder (\path{~/.lv2}, \path{~/.vst}, \path{C:\Program
  Files\Steinberg\VstPlugins\} or the like).

Should the standalone version ever fail to start, you can reset its
settings by deleting \path{trakmeter_stereo.ini} or
\path{trakmeter_surround.ini}.  These files are located in
\path{~/.config} (GNU/Linux) or \path{%appdata%\.config\} (Windows).

\chapter{Controls}
\label{chap:controls}

\section{Reset button}
\label{sec:reset_button}

\begin{wrapfigure}{r}{0.14\linewidth}
\includegraphics[scale=\screenshotscale,clip]{include/images/button_reset_on.png}
\newline \vspace{-0.9\baselineskip}
\includegraphics[scale=\screenshotscale,clip]{include/images/button_reset_off.png}
\end{wrapfigure}

Click on this button to reset all meters.  You can also use it to get
rid of graphical artefacts, because the current skin will be reloaded
and the meters redrawn.

\section{Crest factor button}
\label{sec:crest_factor_button}

\begin{wrapfigure}{r}{0.14\linewidth}
\includegraphics[scale=\screenshotscale,clip]{include/images/button_crest_factor_on.png}
\newline \vspace{-0.9\baselineskip}
\includegraphics[scale=\screenshotscale,clip]{include/images/button_crest_factor_off.png}
\end{wrapfigure}

When this button is pressed, meter readout uses the K-20 scale (crest
factor of \SI{20}{\dB}).  Disengage the button to change to decibels
relative to digital full-scale (crest factor of \SI{0}{\dB}).

\emph{Please note that although this meter uses the K-20 scale, it is
  by no means a K-System meter.}

\newpage %% layout

\section{Transient button}
\label{sec:transient_button}

\begin{wrapfigure}{r}{0.14\linewidth}
\includegraphics[scale=\screenshotscale,clip]{include/images/button_transient_on.png}
\newline \vspace{-0.9\baselineskip}
\includegraphics[scale=\screenshotscale,clip]{include/images/button_transient_off.png}
\end{wrapfigure}

This button changes the RMS meter's ballistics from ``transient'' mode
to ``classic'' mode.

I find that ``transient'' mode with its fast attack and slow release
times is well suited to setting up levels.  It works fine for both
transient audio sources (drums, percussion, piano) and more ``static''
ones (pads, bass and so on).

If you are used to VU meters, however, ``classic'' mode with its
equally slow attack and release times may feel more comfortable to
you.

\emph{\underline{Note:} I don't use ``classic'' mode myself, so you
  may find that the RMS meter scale needs adjusting.  Please don't
  hesitate to notify me, it's really easy to change this.}

\section{Mixing button}
\label{sec:mixing_button}

\begin{wrapfigure}{r}{0.14\linewidth}
\includegraphics[scale=\screenshotscale,clip]{include/images/button_mixing_on.png}
\newline \vspace{-0.9\baselineskip}
\includegraphics[scale=\screenshotscale,clip]{include/images/button_mixing_off.png}
\end{wrapfigure}

You can use traKmeter as a gain plug-in -- just enable this button and
adjust the gain knob which appears.

When the plug-in is closed, its meters aren't updated, so it uses less
system resources.  On slow computers, however, use traKmeter to find
the correct gain and then exchange it against a simple gain plug-in.

Please keep in mind that this setting should only be used for
\textbf{pre-recorded material}.  The gain stage sits \emph{before} the
meter and thus affects the meter's read-out.  So if you apply a
negative gain during recording, your analog input stage might clip
without the meters hitting the red area!

\section{Select a skin}

\begin{wrapfigure}{r}{0.14\linewidth}
\includegraphics[scale=\screenshotscale,clip]{include/images/button_skin_on.png}
\newline \vspace{-0.9\baselineskip}
\includegraphics[scale=\screenshotscale,clip]{include/images/button_skin_off.png}
\end{wrapfigure}

Click on this button to select the currently used traKmeter skin.  You
can also set a default skin that will be loaded when new plug-ins are
instantiated.

\section{Validation button}
\label{sec:validation_button}

\begin{wrapfigure}{r}{0.14\linewidth}
\includegraphics[scale=\screenshotscale,clip]{include/images/button_validate_on.png}
\newline \vspace{-0.9\baselineskip}
\includegraphics[scale=\screenshotscale,clip]{include/images/button_validate_off.png}
\end{wrapfigure}

Click on this button to open the \textbf{validation window} (see
\ref{chap:validation}) which allows you to play an audio file through
traKmeter and dump internal data.  During validation, the button will
light up and clicking it will stop validation early.

\emph{Unfortunately, the underlying JUCE library does not seem to
  support multi-channel audio files.  You may load such audio files
  into your DAW of choice and use a traKmeter plug-in.}

On Linux, dumped data will be written to \path{stderr}, so just start
the traKmeter standalone or your VST host from the shell and watch the
output coming.  On other systems, have a look at your VST host's log
files (I have successfully used Ableton Live for this).  If that
doesn't work, you might have to start either the traKmeter standalone
or your VST host from a debugger.

As a side note, \textbf{SMA(50)} designates the simple moving average
of 50 values, a neat way to emphasise trends and eliminate short-term
fluctuations.

\section{About button}

\begin{wrapfigure}{r}{0.14\linewidth}
\includegraphics[scale=\screenshotscale,clip]{include/images/button_about_on.png}
\newline \vspace{-0.9\baselineskip}
\includegraphics[scale=\screenshotscale,clip]{include/images/button_about_off.png}
\end{wrapfigure}

Clicking on this button will open the \textbf{about window} where you
will be informed about version number, contributors, copyright and the
GNU General Public License.

\section{Display license}

\begin{wrapfigure}{r}{0.15\linewidth}
\includegraphics[scale=\screenshotscale,clip]{include/images/button_gpl_on.png}
\newline \vspace{-0.9\baselineskip}
\includegraphics[scale=\screenshotscale,clip]{include/images/button_gpl_off.png}
\end{wrapfigure}

This button is located in the \textbf{about window} and does not only
advertise that you are using free software licensed under the
\textbf{GNU General Public License} -- when clicked, it will also open
the license's website in your web browser \dots

\chapter{Meters}
\label{chap:meters}

All meters possess a completely flat frequency response.  Meter scales
can be adjusted using the ``Crest factor'' button (see
\ref{sec:crest_factor_button}).

\section{Average level meter}

\begin{wrapfigure}{r}{0.21\linewidth}
\includegraphics[scale=\screenshotscale,clip]{include/images/level_meter_average.png}
\end{wrapfigure}

The average level meter uses an averaging period of \num{1024}
samples.  It has been calibrated according to
\href{http://www.aes.org/publications/standards/search.cfm?docID=21}{AES17-1998}
so that sine wave signals read the same on both peak and average
meters.

Peaks will be held for \SI{10}{\second} and then fall with a speed of
\SI{8.67}{\dB\per\second}.

\subsection{Transient mode}

On rising levels, it takes \SI{10}{\milli\second} for the meter to
reach \SI{99}{\percent} of the final reading.  On falling levels, the
meter switches to a linear fall time of \SI{6}{\dB\per\second}.

\subsection{Classic mode}

Similar to VU meters, it takes \SI{300}{\milli\second} for the meter
to reach \SI{99}{\percent} of the final reading.  This meter exhibits
no overshoot, however.

\section{Peak level meter}

\begin{wrapfigure}{r}{0.21\linewidth}
\includegraphics[scale=\screenshotscale,clip]{include/images/level_meter_peak.png}
\end{wrapfigure}

The peak level meter has a rise time of one sample and a fall time of
\SI{8.67}{\dB\per\second}.  The red LED marked ``OVR'' detects levels
exceeding \SI{-9}{\dBFS} and should never light.

Peaks will be indefinitely held until the meter is reset.

\section{Signal meter}

\begin{wrapfigure}{r}{0.21\linewidth}
\includegraphics[scale=\screenshotscale,clip]{include/images/level_meter_signal.png}
\end{wrapfigure}

The orange signal meter detects peak levels of \SI{-60}{\dBFS} and
above.  It has a rise time of one sample and simply fades out if the
level falls below the threshold.

\chapter{Recording tips}
\label{chap:recording_tips}

Over the years, I have accumulated a couple of recording tips.  You
may not know some of them, so read ahead \dots

\begin{description}

\item[Use a good preamp.]  ``Good'' doesn't mean your preamp has to
  have a lot of channels or features.  To the contrary!  Go for a
  simple design and invest your money in professional quality instead.
  Recordings made with a good preamp make mixing much easier -- the
  tracks simply seem to fall into place.

\item[ Use the preamp's gain control.]  If necessary, crank up the
  preamp to yield the needed output level.  Do not fear the preamp's
  internal noise -- making up for low gain in later stages will likely
  result in even more noise!  Also see \ref{sec:gain_staging}.

\item[Avoid unballanced equipment.]  Run all signals on balanced lines
  with a nominal level of \SI[addsign=all]{+4}{\dBu}.  If you can't,
  use DI boxes or transformers and read the previous sentence again
  \dots

\item[Use short audio chains.]  All equipment adds noise or may
  otherwise degrade audio, so keep your audio recording chains as
  short as possible.

  For example, instead of routing your mixer between preamp and hard
  disk recorder, connect the mixer to your hard disk recorder's
  \emph{outputs}.  This simple change can lead to much better
  recordings (especially with cheap mixers) and you'll still be able
  to hear yourself and other tracks during recording.

\item[Record at lower levels.]  Record digital audio at
  \textbf{\SI{-20}{\dBFS} RMS} with peak levels not exceeding
  \textbf{\SI{-9}{\dBFS}}.  For an in-depth explanation, see
  \ref{sec:gain_staging}.

\item[Record in mono.]  Most audio sources do not contain stereo
  information that is useful in a mixing context (notable exceptions
  are audience recordings, string sections and sometimes pianos).  The
  pseudo-stereo effects of some synthesisers may even cause phasing
  issues in the mixing stage.

  Recording these sources in stereo will only waste space on your hard
  disk and make you miserable during mixing.  So why not record them
  in mono in the first place?

\item[Use high bit depths.]  Do yourself a favour and record at bit
  depths of 24 bit instead of 16 bit.  Although most digital audio
  converters only provide 20 bits of \emph{noise-free} audio, the
  additional bits still provide an incredible amount of extra detail
  and you can record at lower levels without losing information.  When
  properly dithered, changing to a lower bit depth even preserves
  quite a bit of that detail.

  Also, if you edit audio files or apply effects, calculation errors
  are inevitable.  At 24 bit, however, most of these artifacts are
  \SI{48}{\dB} lower in level (and thus inaudible) compared to 16 bit
  audio files.

  Your digital audio workstation's bus should use at least 32 bits
  (floating point) to avoid accumulation of the above-mentioned
  artifacts.

\item[Avoid sample rate conversion.]  Sample rate conversion usually
  degrades audio (especially small changes of a few \si{\kilo\hertz}),
  so try to record at the target sample rate.  For instance, tracking
  for a CD release should be carried out at \SI{44.1}{\kilo\hertz}
  instead of \SI{48}{\kilo\hertz}.

  There are of course exceptions to the rule, for instance you may
  prefer to track on a professional DAT machine (\SI{48}{\kilo\hertz})
  when your only other choice is using a consumer audio interface.

  For tracking at higher sample rates, it pays to use exact multiples
  of your target sample rate (such as \SI{88.2}{\kilo\hertz} instead
  of \SI{96}{\kilo\hertz}) if your hardware and software permit.
  Please note that some professionals actually advise against using
  higher sample rates due to the possible build-up of noise beyond
  \SI{20}{\kilo\hertz}.  It is also much more demanding on your
  computer, audio equipment and plug-ins -- and may not be worth the
  hassle.  Try changing from recording at 16 bit to 24 bit first.

  Finally, only use professional software for sample rate conversion.
  This is by no means a trivial task.

\item[Concentrate on recording.]  When tracking, try to not interfere
  with the flow of the session.  This is easily done by keeping
  editing and mixing to the bare minimum.

  For example, I currently track using an old hard disk recorder, as
  digital audio workstations tend to distract me too much.

\item[Avoid copy'n'paste.]  Quite a lot of today's electronic music
  sounds like (and actually is) one short loop that was ``arranged''
  by occasionally muting some of its tracks.  This takes away all the
  small inaccuracies that happen when humans play instruments.  It
  also makes such tracks sound absolutely lifeless.

  So instead of looping a track, record a couple of takes and comp the
  best ones.  You'll be surprised at the difference it makes!

\item[Do not fix things later.]  A bad recording is a bad recording is
  a bad recording.  You can't really ``fix it in the mix''.  So tools
  like Auto-Tune, extreme EQ or the edit button should be seen as a
  last resort.  It's easy to kill all of a track's vibe in the
  process.

  Instead, record a few more takes.  Treat your room (acoustically and
  in terms of positive vibe).  Experiment with microphone placement.
  Try everything you can to help the musicians perform better.  Maybe
  you even have to look for better musicians \dots

\end{description}

\chapter{Validation}
\label{chap:validation}

I have gone to great lengths to ensure that the meters read correctly.
You want to validate for yourself?  Just download and extract the
source code.  The directory \path{validation} contains instructions
and FLAC-compressed wave files.  A word of warning: these audio files
may \textbf{damage your ears} and speakers, so please watch your
monitor levels!

Begin by starting traKmeter.  If in a Bash shell, try this:

\begin{VerbatimBoth}
  ./trakmeter_stereo 2>&1 | tee /tmp/validate.log
\end{VerbatimBoth}

\begin{wrapfigure}{r}{0.32\linewidth}
\includegraphics[scale=0.60,clip]{include/images/dialog_validation.png}
\end{wrapfigure}

After opening the \textbf{validation window} (see
\ref{sec:validation_button}), click on the ellipsis button (the one
with the dots) to select an audio file for playback through traKmeter.
Please make sure that the sample rates of your host (\textbf{Host SR})
and the audio file match, otherwise the results will not be correct.

Now, select which \textbf{variables} (if any) should be dumped.  You
may also restrict dumped data to a specific audio \textbf{channel}.
Check \textbf{CSV} if you want to feed the output to a parser.

Finally, click on the \textbf{validate} button to reset all meters and
start playback of the selected audio file.  All audio input will be
discarded during playback and for an additional twenty seconds.  To
stop playback early, simply click on the \textbf{validate} button
again.

\section{Validation status}

\begin{minipage}{1.0\linewidth}
  \renewcommand{\thempfootnote}{\arabic{mpfootnote}}
  \begin{tabular}{>{\bfseries}llcc}

    &
    \textbf{Test} &
    \textbf{Valid} \\

    Average level meter &
    visuals &
    \Checkmark{} \\

    &
    readout &
    \Checkmark{} \\

    Peak level meter &
    visuals &
    \Checkmark{} \\

    &
    readout &
    \Checkmark{} \\

    Signal meter &
    visuals &
    \Checkmark{} \\

  \end{tabular}
\end{minipage}

\chapter{Help needed}
\label{chap:help_needed}

As traKmeter was coded using cross-platform code, it should be easy to
compile on Mac OS X.  Unfortunately, I happen to not have a Mac \dots

In case you want to help, please see the next chapter for an email
address.  You’ll need sufficient experience in coding, compiling and
debugging, though, so no beginners please!

\chapter{Final words}
\label{chap:final_words}

I want to thank \textbf{Rickard} of Interfearing Sounds for asking me
how to use K-Meter for tracking.  This question and the following
thoughts really got traKmeter started.  I'd like to thank
\textbf{bram@smartelectronix} for his code to calculate logarithmic
rise and fall times.  I must also thank the \textbf{beta testers} and
\textbf{users of traKmeter} for sending kind words, suggestions and
bug reports.  Finally, I want to thank the \textbf{open source
  community} for making all of this possible.

Although coding traKmeter has been a lot of fun, it has also been a
lot of work.  So if you like traKmeter, why not send me a short email
and tell me so?  Write a few words about yourself, send suggestions
for future updates or volunteer to create a nice skin.  I also really
enjoy listening to music that you may have produced using my
software\dots

Here is my email address (please remove ``\texttt{-nospam}''):

\begin{center}
  \texttt{"Martin Zuther" <code-nospam@mzuther.de>}
\end{center}

Thanks for using free software.  I hope you'll enjoy it!

\begin{wrapfigure}{l}{0.17\linewidth}
  \includegraphics[scale=0.40,clip]{include/images/trademark_vst.png}
\end{wrapfigure}

\emph{VST is a trademark and software of Steinberg Media Technologies
  GmbH.}

\begin{wrapfigure}{l}{0.17\linewidth}
  \includegraphics[scale=0.60,clip]{include/images/trademark_asio.png}
\end{wrapfigure}

\emph{ASIO is a trademark and software of Steinberg Media Technologies
  GmbH.}

\appendix

\chapter{How to build traKmeter}
\label{chap:build_trakmeter}

\section{Preparing GNU/Linux}

To build traKmeter yourself, I recommend setting up a \texttt{chroot}
environment.  This is fast and easy to do on Debian-based systems and
might save you a \textbf{lot} of trouble.  At the time of writing, I'm
using Linux Mint 17, but the procedure should be similar on
your distribution of choice.

Start by installing the necessary packages:

\begin{VerbatimBoth}
  sudo apt-get install debootstrap schroot
\end{VerbatimBoth}

Then install the \texttt{chroot} base system, execute the following
statements (please change \path{http://ftp.de.debian.org/debian/} to a
\href{http://www.debian.org/mirror/list}{mirror} close to you):

\begin{Verbatim32}
  sudo debootstrap --variant=buildd \
    --arch i386 stable \
    /srv/chroot/stable_i386 \
    http://ftp.de.debian.org/debian/
\end{Verbatim32}

\begin{Verbatim64}
  sudo debootstrap --variant=buildd \
    --arch amd64 stable \
    /srv/chroot/stable_amd64 \
    http://ftp.de.debian.org/debian/
\end{Verbatim64}

Running \path{debootstrap} will take some time.  Meanwhile, add the
following lines to \path{/etc/schroot/schroot.conf} (make sure you
remove all preceding white space so that each line begins in the first
column):

\begin{VerbatimBoth}
  [stable-i386]
  description=Debian stable (i386)
  directory=/srv/chroot/stable_i386
  personality=linux32
  root-users=username
  type=directory
  users=username,another_user

  [stable-amd64]
  description=Debian stable (amd64)
  directory=/srv/chroot/stable_amd64
  personality=linux
  root-users=username
  type=directory
  users=username,another_user
\end{VerbatimBoth}

Please make the necessary changes to \texttt{username}.  You may also
add additional users, like \texttt{another\_user}.  If you experience
problems, try to change \texttt{stable} to a release name such as
\texttt{wheezy}.

When \path{debootstrap} is done, log in as superuser:

\begin{Verbatim32}
  schroot -c stable-i386 -u root
\end{Verbatim32}

\begin{Verbatim64}
  schroot -c stable-amd64 -u root
\end{Verbatim64}

You'll have to install a few packages -- \path{less} and \path{vim}
are optional, but might come in handy:

\begin{VerbatimBoth}
  apt-get update
  apt-get -y install bash-completion \
    libasound2-dev libjack-jackd2-dev \
    mesa-common-dev xorg-dev less vim
  apt-get clean
\end{VerbatimBoth}

If you like \path{bash} completion, you might also want to open the
file \path{/etc/bash.bashrc} and unquote these lines:

\begin{VerbatimBoth}
  # enable bash completion in interactive shells
  if [...]
    [a couple of lines...]
  fi
\end{VerbatimBoth}

Finally, log out and log in as normal user:

\begin{Verbatim32}
  schroot -c stable-i386
\end{Verbatim32}

\begin{Verbatim64}
  schroot -c stable-amd64
\end{Verbatim64}

In this \path{chroot} shell, install the dependencies (see below).
Congratulations -- you are now ready to build traKmeter!

\section{Dependencies}

\subsection{premake}
\label{sec:dependencies_premake}

\begin{tabbing}
  \hspace*{6em}\=\=\kill

  Importance:  \> required \\
  Version:     \> 4.3 \\
  License:     \> BSD \\
  Homepage:    \> \href{http://industriousone.com/premake}{industriousone.com/premake}
\end{tabbing}

\subsubsection{Installation}

Place the binary somewhere in your \path{PATH}.  Depending on your
platform, you should run \path{premake} using the scripts
\path{Builds/run_premake.sh} or \path{Builds/run_premake.bat}.

To change the premake file using the provided Jinja templates, install
the necessary dependencies and run the Python script
\path{Builds/create_premake.py}.

\newpage %% layout

\subsection{JUCE library}

\begin{tabbing}
  \hspace*{6em}\=\=\kill

  Importance:  \> required \\
  Version:     \> 4.2.2 \\
  License:     \> ISC and GPL v3 (among others) \\
  Homepage:    \> \href{http://www.juce.com/}{www.juce.com}
\end{tabbing}

\subsubsection{Installation}

Extract the archive into the directory \path{libraries/juce}.

If you want to build the LV2 plug-in, please extract the archive
\path{distrho_lv2-xxxxxxxxxx.tar.gz} into the same directory.

\subsection{Virtual Studio Technology SDK}

\begin{tabbing}
  \hspace*{6em}\=\=\kill

  Importance:  \> optional \\
  Version:     \> 3.6.5 \\
  License:     \> proprietary \\
  Homepage:    \> \href{http://ygrabit.steinberg.de/}{ygrabit.steinberg.de}
\end{tabbing}

\subsubsection{Installation}

Just extract the archive into the directory
\path{libraries/vstsdk3.6.5}.

\subsection{Audio Streaming Input Output SDK}

\begin{tabbing}
  \hspace*{6em}\=\=\kill

  Importance:  \> optional \\
  Version:     \> 2.3 \\
  License:     \> proprietary \\
  Homepage:    \> \href{http://ygrabit.steinberg.de/}{ygrabit.steinberg.de}
\end{tabbing}

\subsubsection{Installation}

Simply extract the archive into the directory
\path{libraries/asiosdk2.3}.

\subsection{Python}

\begin{tabbing}
  \hspace*{6em}\=\=\kill

  Importance:  \> optional \\
  Version:     \> 3.5 (or higher) \\
  License:     \> Python Software Foundation License \\
  Homepage:    \> \href{http://www.python.org/}{www.python.org}
\end{tabbing}

You'll only need Python if you want to change the premake file (see
\ref{sec:dependencies_premake}) using Jinja templates or for building
\num{64}-bit versions of traKmeter using Visual Studio Express.

\subsubsection{Installation (Windows)}

You can download an installer from the website.  Please also install
the \href{http://msdn.microsoft.com/windows/bb980924.aspx}{Windows
  SDK} and change \path{run_premake.bat} to reflect the SDK's version
number.

\subsection{Jinja}

\begin{tabbing}
  \hspace*{6em}\=\=\kill

  Importance:  \> optional \\
  Version:     \> 2.8 (or higher) \\
  License:     \> BSD \\
  Homepage:    \> \href{http://http://jinja.pocoo.org/}{jinja.pocoo.org}
\end{tabbing}

You'll only need Jinja if you want to change the premake file using
templates (see \ref{sec:dependencies_premake}).

\subsection{Artistic Style}

\begin{tabbing}
  \hspace*{6em}\=\=\kill

  Importance:  \> optional \\
  Version:     \> 2.05 \\
  License:     \> LGPL v3 \\
  Homepage:    \> \href{http://astyle.sourceforge.net/}{astyle.sourceforge.net}
\end{tabbing}

This application formats the code so it looks more beautiful and
consistent.  Thus, you only have to install it if you plan to help me
with coding traKmeter.

\subsubsection{Installation}

Place the binary somewhere in your \path{PATH}.  Depending on your
platform, you should run \path{astyle} using the scripts
\path{Source/format_code.sh} or \path{Source/format_code.bat}.

\section{Building on GNU/Linux}

After preparing the dependencies, start your \texttt{chroot}
environment, change into the directory \path{build} and execute

\begin{VerbatimBoth}
  ./run_premake.sh
  make config=CFG TARGET
\end{VerbatimBoth}

where \application{CFG} is one of \application{debug32},
\application{debug64}, \application{release32} and
\application{release64}, and \application{TARGET} is the version you
want to compile, such as \application{linux\_standalone\_stereo}.

The compiled binaries will end up in the directory \path{bin}.

\section{Building on Microsoft Windows}

After preparing the dependencies, change into the directory
\path{build} and execute

\begin{VerbatimBoth}
  ./run_premake.bat
\end{VerbatimBoth}

Then change into the directory \path{Builds/windows/vs20xx}, open the
project file with the corresponding version of Visual C++ and build
the project.

The compiled binaries will end up in the directory \path{bin}.

\chapter{Licenses}

\scriptsize
\input{include/gpl_v3.tex}
\normalsize

\scriptsize
\input{include/cc-by-sa-4.0.tex}
\normalsize

\end{document}


%%% Local Variables:
%%% mode: latex
%%% mode: outline-minor
%%% TeX-command-default: "Rubber"
%%% TeX-PDF-mode: t
%%% ispell-local-dictionary: "british"
%%% End:
